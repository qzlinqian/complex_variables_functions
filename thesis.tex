\documentclass[18pt]{article}  
\usepackage{latexsym}  
\usepackage[fleqn]{amsmath}  
\usepackage{amssymb}
\usepackage{amstext}
\usepackage{xeCJK}
\usepackage{setspace}
\renewcommand{\baselinestretch}{1.5}
\usepackage{geometry}
\geometry{left=3.0cm,right=3.0cm,top=2.5cm,bottom=2.5cm}
\begin{document}
\title{复变函数引论思考题}  
\author { 林祺安 \\  
	机械工程系 \\}  
\date{\today}  
\maketitle
\section{} 设$f(z)=z^{n}+\sum\nolimits_{k=1}^{n}c_{k}z^{n-k}$。证明:$f(z)$为分圆多项式的充要条件是$f(z)$的n个零点为圆内接正n边形的n个顶点。\\


\begin{itemize}
\item 法一
\end{itemize}

\textbf{充分性:}\\
$\mbox{对分圆多项式} f(z)=z^{n}+c_{n} \text{,令} f(z)=0 \text{,则可得} f(z) \text{的零点为\\}
z=z_{k}=|c_{n}|^{\frac{1}{n}}\cdot e^{i\cdot\frac{arg(-c_{n})+2(k-1)\pi}{n}},k=1,2,\cdots,n$\\
易知
\begin{flalign*}
|z_{k}| & =\sqrt{z_{K}\cdot \overline{z_{k}}}\\
&=\sqrt{|c_{n}|^{\frac{1}{n}}\cdot e^{i\cdot\frac{arg(-c_{n})+2(k-1)\pi}{n}}\cdot |c_{n}|^{\frac{1}{n}}\cdot e^{-i\cdot\frac{arg(-c_{n})+2(k-1)\pi}{n}}}\\
&=|c_{n}|^{\frac{1}{n}}
\end{flalign*}
这n个零点全部在以原点为圆心,半径为$|c_{n}|^{\frac{1}{n}}$的圆上。即这n个零点构成一个圆内接n边形。
\begin{align*}
\therefore z_{k+1}-z_{k} &=|c_{n}|^{\frac{1}{n}}\cdot e^{i\cdot\frac{arg(-c_{n})+2k\pi}{n}}-|c_{n}|^{\frac{1}{n}}\cdot e^{i\cdot\frac{arg(-c_{n})+2(k-1)\pi}{n}} \\
&=|c_{n}|^{\frac{1}{n}}\cdot e^{i\cdot\frac{arg(-c_{n})+2(k-1)\pi}{n}} \cdot (e^{\frac{2\pi}{n}}-1)
\end{align*}
\begin{align*}
\therefore |z_{k+1}-z_{k}|^{2} &=|c_{n}|^{\frac{2}{n}} \cdot (e^{\frac{2\pi}{n}}-1)^{2} \cdot e^{i\cdot\frac{arg(-c_{n})+2(k-1)\pi}{n}} \cdot e^{-i\cdot\frac{arg(-c_{n})+2(k-1)\pi}{n}} \\
&=|c_{n}|^{\frac{2}{n}} \cdot (e^{\frac{2\pi}{n}}-1)^{2}, k=1,2,\dots ,n
\end{align*}
(定义$z_{n+1}=|c_{n}|^{\frac{1}{n}}\cdot e^{i\cdot\frac{arg(-c_{n})}{n}}=|c_{n}|^{\frac{1}{n}}\cdot e^{i\cdot\frac{arg(-c_{n})+2(k-1)\pi}{n}}=z_{1}$)\\
$\therefore$ $f(z)$的n个零点构成的n边形的边长均相等\\
又 $\because$ 此n边形为圆内接n边形\\
$\therefore$ $f(z)$的n个零点构成一个正n边形\\
\\
\textbf{必要性:}\\
对正n边形的n个顶点\\
$z=z_{k}=r\cdot e^{i \cdot (\alpha_{0}+ \frac{2(k-1)\pi}{n})}, k=1,2,\cdots, n$\\
有$z_{k}^{n}=r^{n}\cdot e^{i\cdot(n\alpha_{0}+2(k-1)\pi)}, \forall k=1,2,\cdots,n$\\
$\therefore z_{1},z{2},\cdots,z_{n}$均为首一多项式$f(z)=z^{n}-r^{n}\cdots e^{in\alpha_{0}}$的零点\\
又 $z_{1},z_{2},\cdots,z_{n}$互不相等,$degf(z)=n$\\
$\therefore z_{1},z_{2},\cdots,z_{n}$为$f(z)$的全部零点\\
又 $\because f(z)$为首一多项式
$\therefore f(z)=\prod_{k=1}^{n}(z-z_{k})$\\
所以有$f(z)=\prod_{k=1}^{n}(z-z_{k})=z^{n}-r^{n}\cdots e^{in\alpha_{0}}$ \quad \rule{3mm}{3mm}\\
\\
\\
\\
\begin{itemize}
	\item 法二
\end{itemize}
\textbf{充分性:}\\
当$f(z)$为分圆多项式时,有$f(z)=z^{n}+c_{n}$\\
其零点为$z=z_{k}=|c_{n}|^{\frac{1}{n}}\cdot e^{i\cdot\frac{arg(-c_{n})+2(k-1)\pi}{n}},k=1,2,\cdots,n$\\
\begin{align*}
\therefore z_{k+1}-z_{k} &=|c_{n}|^{\frac{1}{n}}\cdot e^{i\cdot\frac{arg(-c_{n})+2k\pi}{n}}-|c_{n}|^{\frac{1}{n}}\cdot e^{i\cdot\frac{arg(-c_{n})+2(k-1)\pi}{n}} \\
&=|c_{n}|^{\frac{1}{n}}\cdot e^{i\cdot\frac{arg(-c_{n})+2(k-1)\pi}{n}} \cdot (e^{\frac{2\pi}{n}}-1)
\end{align*}\\
同理,
\begin{align*}
z_{k+2}-z_{k+1}=|c_{n}|^{\frac{1}{n}}\cdot e^{i\cdot\frac{arg(-c_{n})+2k\pi}{n}} \cdot (e^{\frac{2\pi}{n}}-1)
\end{align*}
(定义$z_{n+t}=z_{t},\forall t \in \mathbb{Z}$)
\begin{align*}
	arg(z_{k+2}-z_{k+1})-arg(z_{k+1}-z_{k})&=\frac{1}{n}\cdot (arg(-c_{n})+2k\pi)-\frac{1}{n}\cdot (arg(-c_{n})+2(k-1)\pi)\\
	&=\frac{2\pi}{n}
\end{align*}
故相邻边夹角为$\pi-\frac{2\pi}{n}=\frac{n-2}{n}\pi$为常数\\
所以\quad$z_{1},z_{2},\cdots,z_{n}$构成的n边形各个顶角相等\\
又\quad 此n边形为圆内接n边形\\
所以 $z_{1},z_{2},\cdots,z_{n}$构成正n边形\\
\\
\textbf{必要性:}\\
对正n边形的n个顶点,以正n边形中心为原点,有
$z_{k+1}=z_{k}\cdot e^{\frac{2\pi}{n}i},k=1,2,\cdots,n$\\
(定义$z_{n+1}=z_{1}$)
$\therefore z_{k+1}^{n}=z_{k}^{n}\cdot e^{2\pi i}=z_{k}^{n},\forall k=1,2,\cdots,n$\\
$\therefore z_{k}^{n}=z_{1}^{n},\forall k=1,2,\cdots,n$\\
令$f(z)=z^{n}-z_{1}^{n}$为分圆多项式,则$z_{1},z_{2},\cdots,z_{n}$互不相同,且皆为$f(z)$的零点\\
又\quad $degf(z)=n$\\
$\therefore z_{1},z_{2},\cdots,z_{n}$为$f(z)$的全部零点\\
$\therefore f(z)=\prod_{k=1}^{n}(z-z_{k})=z^{n}-z_{1}^{n}$ \quad \rule{3mm}{3mm}\\

\section{}
证明:$\zeta(2n)=c_{n}\zeta^{n}(2)=c_{n}(\frac{\pi^2}{6})^{n},\quad c_{1}=1\\
(n+\frac{1}{2})c_{n}=\sum\limits_{k=1}^{n-1}c_{k}c_{n-k},\quad n \geq 2$\\
\\
由于$ln(\frac{sinz}{z})=\sum\limits_{n=1}^{+\infty}ln(1-\frac{z^2}{n^2 \pi^2})$,由Taylor展开,
\begin{equation}
\begin{split}
	ln\frac{sinz}{z} &=\sum\limits_{n=1}^{+\infty} \sum\limits_{k=1}^{+\infty} \frac{(-1)^{k-1} \cdot (-z^2)^k}{k \cdot (n^2 \pi^2)^k}\\
					 &=-\sum\limits_{k=1}^{+\infty} \frac{z^{2k}}{k \pi^{2k}} \sum\limits_{n=1}^{+\infty} \frac{1}{n^{2k}}\\
					 &=-\sum\limits_{k=1}^{+\infty} \frac{z^{2k}}{k \pi^{2k}} \cdot \zeta(2k) \label{basic}
\end{split}
\end{equation}
对上式求导得,
$cotz-\frac{1}{z}=\sum\limits_{k=1}^{+\infty}\frac{-2}{\pi^{2k}}\zeta(2k)\cdot z^{2k-1}$\\
故有\begin{equation}
	zcotz-1=-2\sum\limits_{k=1}^{+\infty}\frac{1}{\pi^{2k}}\zeta(2k) \cdot z^{2k} \label{use_in_3}
\end{equation}
用$\pi z$代替$z$,有
$cot(\pi z)-\frac{1}{\pi z}=-2 \sum\limits_{k=1}^{+\infty}\zeta(2k)\cdot z^{2k-1} \cdot \pi^{-1}$\\
所以有
\begin{equation}
	\pi z \cdot cot(\pi z)-1=-2 \sum\limits_{k=1}^{+\infty}\zeta(2k) \cdot z^{2k}\label{general}
\end{equation}
对\eqref{general}求导得,$\pi cot(\pi z)+\pi z \cdot [cot(\pi z)]'=-2\sum\limits_{k=1}^{+\infty} 2n \cdot \zeta(2k) \cdot z^{2k-1}$ \\
又 $\because \quad [cot(\pi z)]'=\pi(-1-cot^2(\pi z))$ \\
$\therefore \pi cot(\pi z)+\pi^2 z \cdot cot'(\pi z)= \pi cot(\pi z)-\pi^2 z-\frac{1}{z} \cdot (\pi z \cdot cot(\pi z))^2$\\
将\eqref{general}代入上式可得,\\
$-2\sum\limits_{k=1}^{+\infty} 2n \cdot \zeta(2k) \cdot z^{2k-1}=\frac{1}{z} [-2\sum\limits_{k=1}^{+\infty}\zeta(2k) \cdot z^{2k}+1]-\pi^2 z-\frac{1}{z} \cdot [-2\sum\limits_{k=1}^{+\infty}\zeta(2k) \cdot z^{2k}+1]^2$\\
所以有
\begin{align*}
	2\sum\limits_{k=1}^{+\infty}(2n-1)\cdot \zeta(2k) \cdot z^{2k}+1 &=\pi^2 z^2+[-2\sum\limits_{k=1}^{+\infty}\zeta(2k) \cdot z^{2k}+1]^2\\
	&=\pi^2 z^2+4(\sum\limits_{k=1}^{+\infty}\zeta(2k) \cdot z^{2k})^2-4\sum\limits_{k=1}^{+\infty}\zeta(2k) \cdot z^{2k}+1
\end{align*}
整理得,
\begin{equation}
	\begin{split}
		\sum\limits_{k=1}^{+\infty}(2n+1)\cdot \zeta(2k) \cdot z^{2k-1} &=2(\sum\limits_{k=1}^{+\infty}\zeta(2k) \cdot z^{2k})^2+\pi^2 z^2 \\
		&=2\sum\limits_{n=1}^{+\infty}\sum\limits_{k=1}^{n-1}\zeta(2k)\cdot\zeta(2n-2k) z^{2n}+\pi^2 z^2 \label{final}
	\end{split}
\end{equation}
比较等式\eqref{final}两边$2n(n\geq2)$次项的系数,有\\
$(2n+1)\zeta(2n)=2\sum\limits_{k=1}^{n-1}\zeta(2k)\cdot\zeta(2n-2k)$\\
又 \quad$c_{n}=\frac{\zeta(2n)}{\zeta^n(2)}$\\
$\therefore (n+\frac{1}{2})c_{n}=\sum\limits_{k=1}^{n-1}c_{k}\cdot c_{n-k}$ \quad \rule{3mm}{3mm}\\
\\
\\
\\
\section{}
证明:$\frac{\zeta(2n)}{\pi^{2n}}-\frac{\zeta(2n-2)}{3!\pi^{2n-2}}+\frac{\zeta(2n-4)}{5!\pi^{2n-4}}+ \cdots+(-1)^(n-1)\frac{\zeta(2)}{\pi^{2}}+\frac{(-1)^n n}{(2n+1)!}=0, \quad n\in mathbb{N}$\\
将式\eqref{use_in_3}两边同乘以$sinz$得:
\begin{equation}
\begin{split}
	zcosz-sinz &=-2sinz(\sum\limits_{k=1}^{+\infty}\frac{1}{\pi^{2k}}\zeta(2k) \cdot z^{2k})\\
			   &=-2(\sum\limits_{k=1}^{+\infty})(\sum\limits_{k=1}^{+\infty}\frac{1}{\pi^{2k}}\zeta(2k) \cdot z^{2k})\\
			   &=-2\sum\limits_{n=1}^{+\infty} \sum\limits_{k=1}^{n} \frac{(-1)^{n-k}}{\pi^{2k} \cdot (2n-2k+1)!} \zeta(2k) z^{2n+1}
			   \label{fisrt}
\end{split}
\end{equation}
而
\begin{equation}
\begin{split}
	zcosz-sinz &=z(\sum\limits_{n=0}^{+\infty} \frac{(-1)^n z^{2n}}{(2n)!}) -\sum\limits_{k=1}^{+\infty}\frac{(-1)^{k-1}z^{2k-1}}{(2k-1)!}\label{second}\\
			   &=(\sum\limits_{n=1}^{+\infty}\frac{(-1)^n}{(2n)!} - \sum\limits_{n=1}^{+\infty} \frac{(-1)^n}{(2n+1)!})z^{2n+1}
\end{split}
\end{equation}
比较式\eqref{fisrt}和式\eqref{second},可以得到
\begin{equation}
	-2\sum\limits_{n=1}^{+\infty} \sum\limits_{k=1}^{n} \frac{(-1)^{n-k}}{\pi^{2k} \cdot (2n-2k+1)!} \zeta(2k) z^{2n+1}=(\sum\limits_{n=1}^{+\infty}\frac{(-1)^n}{(2n)!}  \frac{(-1)^n}{(2n+1)!})z^{2n+1} \label{final3}
\end{equation}
比较式\eqref{final3}两端的2n+1次项系数,可得
\begin{align*}
-2\sum\limits_{k=1}^{n} \frac{(-1)^{n-k}}{\pi^{2k} \cdot (2n-2k+1)!} \zeta(2k) &= \frac{(-1)^n}{(2n)!}  \frac{(-1)^n}{(2n+1)!} \\
     &= \frac{(-1)^n \cdot 2n}{(2n+1)!}
\end{align*}
整理得,$\sum\limits_{k=0}^{n-1} \frac{(-1)^k \zeta(2n-2k)}{(2k+1)! \pi^(2n-2k)} + \frac{(-1)^n \cdot n}{(2n+1)!} =0$ \quad \rule{3mm}{3mm}\\
\end{document}